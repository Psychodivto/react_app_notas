\documentclass{article}
\usepackage[utf8]{inputenc}
\usepackage{graphicx}
\usepackage{ragged2e}

\title{Aplicacion react (Notas)}
\author{Diego Ivan Pérez Conde }
\date{June 2022}

\begin{document}

\maketitle

\section{Introduction}

\justifying

\textbf{EL proyecto por el cual base mi aplicacion se trata de una sencilla aplicacion web que guarda notas ya sea para agregar documentos, imagenes, fotos, etc...}

\textbf{Donde de manera rapida el usuario pueda guardar lo anterior mencionado asi como puede agregar texto para ya sea guardar una nota rapido o un recordatorio, que se guardara en firebase como mi base de datos y almacenara la información y asi mismo pueda guardar archivos de manera segura. }

\subsection{Tecnologias usadas}

\justifying{Las tecnologias que se aplicaron para la elaboracion de la aplicacion fueron:}\\

\justifying{React:Que es la libreria que se ocupo de javascript para poder hacer la aplicacaion web muy similar a lo que puede ser angula pero más facil, que lo podria hacer de un elaboracion más rapida para hacer practicas.}\\

\justifying{Firebase:Nuestra base de datos para poder guardar asi mismo los datos de nuestra aplicacion que se hacer manera mas facil, y ayuda al usuario.}\\

\justifying{Autenticacion:La usamos para poder hacer el ingreso de aplicaciones de terceros como los servicios de google, facebook entre otros. tambien nos ayudo para el hacer el ingreso de nuestra app fuera más facil sin la necesidad de tener que ingresar un correo o registrarse. }\\

\justifying{Storage:Ocupe store para poder almacenar la informacion de cada uno de mis usuarios que se registran en mi app o se loguearan con los servicios de google entre otros, para poder guardar su informacion como imagenes musica pdf's, word etc.. }\\

\justifying{Firestore database:Aqui se recopila la informacion de mis usuarios que se logguen como su correo y contraseñas que los almacenare en mi base de datos que haya creado.}\\

\subsection{Conclusión}\\

\justifying{La parte del desarrollo más dificil de la aplicacion fue la elaboracion del CRUD para poder obtener datos de mis usuarios ya que no podia agregarlos asi mismo para cuando queria guardar informacion no podia pero resolviendo mis metodos de authenticacion que tenia en mi codigo pude arreglarlo, con ello aprendi a realizar de manera rapida un metodo para poder guardar datos con metodos de authenticacion.}\\

\includegraphics[scale=.5]{gracias.jpg}








\end{document}
